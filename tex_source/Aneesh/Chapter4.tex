% ---------
%  Compile with "pdflatex hw0".
% --------
%!TEX TS-program = pdflatex
%!TEX encoding = UTF-8 Unicode

\documentclass[11pt]{article}
\usepackage{jeffe,handout,graphicx}
\usepackage[utf8]{inputenc}     % Allow some non-ASCII Unicode in source
\usepackage{amsmath}
\usepackage{listings}
\usepackage{tikz}

%  Redefine suits
\usepackage{pifont}
\def\Spade{\text{\ding{171}}}
\def\Heart{\text{\textcolor{Red}{\ding{170}}}}
\def\Diamond{\text{\textcolor{Red}{\ding{169}}}}
\def\Club{\text{\ding{168}}}

\def\Cdot{\mathbin{\text{\normalfont \textbullet}}}
\def\Sym#1{\textbf{\texttt{\color{BrickRed}#1}}}
\let\oriinfty=\infty
\def\infty{{\oriinfty\llap{\vrule height.24em depth-.2em width.2em}}}

%%% magic code starts
\mathcode`*=\string"8000
\begingroup
\catcode`*=\active
\xdef*{\noexpand\textup{\string*}}
\endgroup
%%% magic code ends
%%% check mark
\def\checkmark{\tikz\fill[scale=0.4](0,.35) -- (.25,0) -- (1,.7) -- (.25,.15) -- cycle;}

% =====================================================
%   Define common stuff for solution headers
% =====================================================
\Class{Abstract Algebra}
\Semester{}
\Authors{1}
\AuthorOne{[Your name here]}
%\Section{}

% =====================================================
\begin{document}

% Copy this section again for a new solution
% ---------------------------------------------------------
\GenericHeader{Proposition}{4}{12} % replace Chapter with Exercise for a different header
\begin{solution} \hfill \\
In a group $G=<a>$ of order $n$, $a^k=e$ iff $n|k$:
Proof:
By the division theorem:
$$k = m(n)+r, \quad 0\leq r<n, m\in \mathbb{Z}^+$$
$$\Rightarrow a^k = a^{m(n)+r}$$
$$\Rightarrow a^k = a^{m(n)}a^{r}$$
$$\Rightarrow a^k = e^{m}a^{r}$$
$$\Rightarrow a^k = e\cdot a^{r}$$
$$\Rightarrow e = e\cdot a^{r}$$
$$\Rightarrow e = a^{r}$$

However, $r<n$ and $n$ is the least positive integer such that $a^n=e$, $\therefore$ $r = 0$ and $n|k$.
\end{solution}
\vspace{6pt}
\hrule
\vspace{6pt}
% ---------------------------------------------------------

% Copy this section again for a new solution
% ---------------------------------------------------------
\ExerciseHeader{4}{4} % replace Chapter with Exercise for a different header
\begin{itemize}
    \item[1)] Prove or disprove each of the following statements.
    \begin{itemize}
        \item [a)]All of the generators of $\mathbb{Z}_{60}$ are prime.
        \item [b)] $U(8)$ is cyclic.
        \item [c)]$\mathbb{Q}$ is cyclic.
        \item [d)]If every proper subgroup of a group $G$ is cyclic, then $G$ is a cyclic group.
        \item [e)]A group with a finite number of subgroups is finite.
        \begin{solution} \hfill \\
            We know that for any finite group, there are a finite number of subgroups upper bounded by the cardinality of the powerset of the group.\\
            Therefore, we just need to show that there are no infinite groups with finite proper subgroups. Suppose $G$ is an infinite group with a finite collection of proper subgroups $S_G$.
        \end{solution}

    \end{itemize}
    \item[46)] Prove that $z=\cos\theta + i\cdot \sin\theta$ has infinite order for some $\theta \in \mathbb{Q}$\\
    \begin{solution} \hfill \\
        To avoid ambiguity let $\e$ be the identity element.
        $$z=e^{i\theta}, \quad \e=1=e^{2i\pi}$$
        We want that for some $k\in \mathbb{Z}^+$, $z^k=\e=e^{2ni\pi}$ for some $n\in \mathbb{Z}^+$. This gives us:
        $$ik\theta=2ni\pi$$
        $$k\theta=2n\pi$$
        $$\frac{k}{2n}\theta=\pi$$
        Since $\frac{k}{2n}$ and $\theta \in \mathbb{Q}$, $\frac{k}{2n}\theta \in \mathbb{Q}$. However, $\pi \notin \mathbb{Q}$, contradiction,
        hence the order of $z$ is $\infty$.
    \end{solution}
\end{itemize}
\vspace{6pt}
\hrule
\vspace{6pt}
% ---------------------------------------------------------
\end{document}

