% ---------
%  Compile with "pdflatex hw0".
% --------
%!TEX TS-program = pdflatex
%!TEX encoding = UTF-8 Unicode

\documentclass[11pt]{article}
\usepackage{jeffe,handout,graphicx}
\usepackage[utf8]{inputenc}     % Allow some non-ASCII Unicode in source
\usepackage{amsmath}
\usepackage{listings}
\usepackage{tikz}

%  Redefine suits
\usepackage{pifont}
\def\Spade{\text{\ding{171}}}
\def\Heart{\text{\textcolor{Red}{\ding{170}}}}
\def\Diamond{\text{\textcolor{Red}{\ding{169}}}}
\def\Club{\text{\ding{168}}}

\def\Cdot{\mathbin{\text{\normalfont \textbullet}}}
\def\Sym#1{\textbf{\texttt{\color{BrickRed}#1}}}
\let\oriinfty=\infty
\def\infty{{\oriinfty\llap{\vrule height.24em depth-.2em width.2em}}}

%%% magic code starts
\mathcode`*=\string"8000
\begingroup
\catcode`*=\active
\xdef*{\noexpand\textup{\string*}}
\endgroup
%%% magic code ends
%%% check mark
\def\checkmark{\tikz\fill[scale=0.4](0,.35) -- (.25,0) -- (1,.7) -- (.25,.15) -- cycle;}

% =====================================================
%   Define common stuff for solution headers
% =====================================================
\Class{Abstract Algebra}
\Semester{}
\Authors{1}
\AuthorOne{Aneesh Durg}
%\Section{}

% =====================================================
\begin{document}

% Copy this section again for a new solution
% ---------------------------------------------------------
\GenericHeader{Proposition}{4}{12} % replace Chapter with Exercise for a different header
\begin{solution} \hfill \\
In a group $G=<a>$ of order $n$, $a^k=e$ iff $n|k$:
Proof:
By the division theorem:
$$k = m(n)+r, \quad 0\leq r<n, m\in \mathbb{Z}^+$$
$$\Rightarrow a^k = a^{m(n)+r}$$
$$\Rightarrow a^k = a^{m(n)}a^{r}$$
$$\Rightarrow a^k = e^{m}a^{r}$$
$$\Rightarrow a^k = e\cdot a^{r}$$
$$\Rightarrow e = e\cdot a^{r}$$
$$\Rightarrow e = a^{r}$$

However, $r<n$ and $n$ is the least positive integer such that $a^n=e$, $\therefore$ $r = 0$ and $n|k$.
\end{solution}
\vspace{6pt}
\hrule
\vspace{6pt}
% ---------------------------------------------------------

% Copy this section again for a new solution
% ---------------------------------------------------------
\GenericHeader{Scratch}{4}{0} % replace Chapter with Exercise for a different header
Prove that numbers that are not relatively prime cannot be generators
\begin{solution} \hfill \\
    Suppose $n=n'p$, $k=k'p$. Where $p < n$ is prime, $0< n'<n$ and $0<k'<k$\\
    Let's consider $<k>$.
    $$mk=p(mk')$$
    Suppose $<k'>$ is not a generator for $\mathbb{Z}_n$ and contains $z$ values, then $p(mk')$ can take at most $z$ values.\\
    Suppose $<k'>$ is a generator for $\mathbb{Z}_n$, then it remains to be seen if $<p>$ is also a generator. However, $pn'=0_n$ and since  $0<n'<n$, $<p>$ can have at most $n'$ values.
\end{solution}
\vspace{6pt}
\hrule
\vspace{6pt}
Prove that numbers that are relatively prime must be generators
\begin{solution} \hfill \\
    Need to show that if $\forall$ primes $p$ such that $p|n$, $p\nmid r$ then $<r> = \mathbb{Z}_n$. We know that $r=p_1\cdot p_2\cdot p_3\dots$ for some sequence of primes. However, we just need to show that this is true for the case where $r=p$ for any prime $p\nmid n$. Once we show this, given $k$ primes (not necessarily unique) we know that $<p_1>=<p_2>=\dots =<p_k>=<\Pi_{m=1}^{k} p_m>=\mathbb{Z}_n$\\
    Suppose for some $m<n$, $mp=_n 0$ (which is to say that $p$ is not a generator). Then $mp=n$ and thus $p\mid n$, contradiction, $<p>=\mathbb{Z}_n$
\end{solution}
\vspace{6pt}
\hrule
\vspace{6pt}
% ---------------------------------------------------------

% Copy this section again for a new solution
% ---------------------------------------------------------
\ExerciseHeader{4}{4} % replace Chapter with Exercise for a different header
\begin{itemize}
    \item[1)] Prove or disprove each of the following statements.
    \begin{itemize}
        \item [a)]All of the generators of $\mathbb{Z}_{60}$ are prime.
        \begin{solution}
            True. Only numbers that are relatively prime can be a generator (see scratch 4.0 above), no composite numbers are relatively prime to 60 and less than 60, with the first relatively prime number being 77.
        \end{solution}
        \item [b)] $U(8)$ is cyclic.
        \item [c)]$\mathbb{Q}$ is cyclic.
        \begin{solution}
            False. Suppose $x$ is a generator. $\exists y\in [x,x+x]$ therefore $x$ cannot generate $y$.
        \end{solution}
        \item [d)]If every proper subgroup of a group $G$ is cyclic, then $G$ is a cyclic group.
        \begin{solution}
            False. Consider $S_3$.
        \end{solution}
        \item [e)]A group with a finite number of subgroups is finite.
        \begin{solution}
            We know that for any finite group, there are a finite number of subgroups upper bounded by the cardinality of the powerset of the group.\\
            Therefore, we just need to show that there are no infinite groups with finite proper subgroups. Suppose $G$ is an infinite group with a finite collection of proper subgroups $S_G$.
        \end{solution}
    \end{itemize}
    \vspace{6pt}
    \hrule
    \vspace{6pt}
    \item[2)] Find the order of each of the following elements.
    \begin{itemize}
        \item[a)] $5\in \mathbb{Z}_{12}$
        \begin{solution}
            $12$
        \end{solution}
        \item[b)] $\sqrt{3}\in \mathbb{R}$
        \begin{solution}
            $\infty$
        \end{solution}
        \item[c)] $\sqrt{3}\in \mathbb{R}^*$
        \begin{solution}
            $\infty$
        \end{solution}
        \item[d)] $-i\in \mathbb{C}^*$
        \begin{solution}
            $4$
        \end{solution}
        \item[e)] $72 \in \mathbb{Z}_{240}$
        \begin{solution} \hfill \\
            \texttt{ [(72*i)\%240 for i in range(240)].index(0, 1) \# = 10}
        \end{solution}
        \item[f)] $312 \in \mathbb{Z}_{471}$
        \begin{solution} \hfill \\
            \texttt{ [(312*i)\%417 for i in range(417)].index(0, 1) \# = 139}
        \end{solution}
    \end{itemize}
    \vspace{6pt}
    \hrule
    \vspace{6pt}
    \item[24)] Number of generators for $\mathbb{Z}_{pq}$?
    \begin{solution}
        $pq - \lfloor\frac{pq}{p}\rfloor - \lfloor\frac{pq}{q}\rfloor$
    \end{solution}
    \vspace{6pt}
    \hrule
    \vspace{6pt}
    \item[31)] Let $G$ be an abelian group. Show that the elements of finite order in $G$ form a subgroup. This subgroup is called the \textbf{torsion subgroup} of $G$.
    \begin{solution} \hfill \\
        Let $G=<S,*>$ and $S'$ be the subset of $S$ of elements with finite order. We first need to prove that $S'$ is closed under $*$. For some $a, b\in G$, we have $a*b=c$.\\
        Suppose $a^n=e$ and $a\b^m=e$. Then we have that
        $$c^{nm}=(ab)^{nm}=\left(a^{nm}*b^{nm}\right)=e^m*e^n=e$$
        Thus $c$ has order at most $nm$ and thus $c\in S'$.\\
        Since $G$ was a group, $*$ is still associative over $S'$, the identity element is trivially of finite order and thus the only condition that remains to be satisfied is that every element has an inverse in $S'$.
        Suppose $a^k=e$, this can be rewritten as $a*a^k=e$
        $$\Rightarrow a^k=a^{-1}$$
        Some inductive reasoning shows that:
        $$\Rightarrow e=a^{-k}$$
        $$\Rightarrow \left(a^{-1}\right)^k = e$$
        This implies that $a^{-1}\in S'$ and thus $<S', *>$ is a group.
    \end{solution}
    \vspace{6pt}
    \hrule
    \vspace{6pt}
    \item[36)] Prove that the generators of $\mathbb{Z}_n$ are the integers $r$ such that $1\leqr<n$ and $gcd(r,n)=1$.
    \begin{solution} See scratch 4.0
    \end{solution}
    \vspace{6pt}
    \hrule
    \vspace{6pt}
    \item[46)] Prove that $z=\cos\theta + i\cdot \sin\theta$ has infinite order for some $\theta \in \mathbb{Q}$\\
    \begin{solution} \hfill \\
        To avoid ambiguity let $\e$ be the identity element.
        $$z=e^{i\theta}, \quad \e=1=e^{2i\pi}$$
        We want that for some $k\in \mathbb{Z}^+$, $z^k=\e=e^{2ni\pi}$ for some $n\in \mathbb{Z}^+$. This gives us:
        $$ik\theta=2ni\pi$$
        $$k\theta=2n\pi$$
        $$\frac{k}{2n}\theta=\pi$$
        Since $\frac{k}{2n}$ and $\theta \in \mathbb{Q}$, $\frac{k}{2n}\theta \in \mathbb{Q}$. However, $\pi \notin \mathbb{Q}$, contradiction,
        hence the order of $z$ is $\infty$.
    \end{solution}
\end{itemize}
\vspace{6pt}
\hrule
\vspace{6pt}
% ---------------------------------------------------------
\end{document}

