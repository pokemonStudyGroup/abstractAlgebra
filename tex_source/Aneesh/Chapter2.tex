% ---------
%  Compile with "pdflatex hw0".
% --------
%!TEX TS-program = pdflatex
%!TEX encoding = UTF-8 Unicode

\documentclass[11pt]{article}
\usepackage{jeffe,handout,graphicx}
\usepackage[utf8]{inputenc}     % Allow some non-ASCII Unicode in source
\usepackage{amsmath}
\usepackage{listings}
\usepackage{tikz}

%  Redefine suits
\usepackage{pifont}
\def\Spade{\text{\ding{171}}}
\def\Heart{\text{\textcolor{Red}{\ding{170}}}}
\def\Diamond{\text{\textcolor{Red}{\ding{169}}}}
\def\Club{\text{\ding{168}}}

\def\Cdot{\mathbin{\text{\normalfont \textbullet}}}
\def\Sym#1{\textbf{\texttt{\color{BrickRed}#1}}}
\let\oriinfty=\infty
\def\infty{{\oriinfty\llap{\vrule height.24em depth-.2em width.2em}}}

%%% magic code starts
\mathcode`*=\string"8000
\begingroup
\catcode`*=\active
\xdef*{\noexpand\textup{\string*}}
\endgroup
%%% magic code ends
%%% check mark
\def\checkmark{\tikz\fill[scale=0.4](0,.35) -- (.25,0) -- (1,.7) -- (.25,.15) -- cycle;}

% =====================================================
%   Define common stuff for solution headers
% =====================================================
\Class{Abstract Algebra}
\Semester{}
\Authors{1}
\AuthorOne{Aneesh Durg}
%\Section{}

% =====================================================
\begin{document}

% Copy this section again for a new solution
% ---------------------------------------------------------
\ExerciseHeader{2}{3}
Prove that $10^{n+1} + 3*10^n + 5$ is divisible $9$.
\begin{solution} \hfill \\
    For $n=1$:\\
    $$10^2+3*10+5=135$$
    $$9|135\textrm{ is true}$$
    For all $n=0,1,\dots,m$ we assume $(10^{n+1} + 3*10^n + 5) = 9*k$ for some $k$.\\
    For $n=m+1$:
    $$10^{m+2}+3*10^{m+1}+5 $$
    $$= (9+1)10^{m+2}+(9+1)10^{m+1}+5$$
    $$= 9(10^{m+2})+9(10^{m+1})+10^{m+1} + 3*10^m+5$$
    By the inductive hypothesis:
    $$= 9(10^{m+2})+9(10^{m+1})+9*k$$
    $$= 9(10^{m+2}+10^{m+1}+k)$$
\end{solution}
\vspace{6pt}
\hrule
\vspace{6pt}
% ---------------------------------------------------------

% Copy this section again for a new solution
% ---------------------------------------------------------
\ChapterHeader{2}{8}
Prove the Leibniz rule: $(fg)^{(n)}(x) = \Sum^n_{k=0}\binom{n}{k}f^{(k)}(x)g^{(n-k)}(x)$ using mathematical induction
\begin{solution} \hfill \\
    I leave the proof for this being true for $n=1$ as an exercise to the reader.\\
    For all $n=0,1,\dots,m$ we assume $(fg)^{(n)}(x) = \Sum^n_{k=0}\binom{n}{k}f^{(k)}(x)g^{(n-k)}(x)$.\\
    For $n=m+1$:\\
    $$(fg)^{(m+1)}(x) = \frac{d}{dx}\left(\Sum^m_{k=0}\binom{m}{k}f^{(k)}(x)g^{(m-k)}(x)\right)$$
    $$= \Sum^m_{k=0}\binom{m}{k}\left(f^{(k+1)}(x)g^{(m-k)}(x) + f^{(k)}(x)g^{(m-k+1)}(x)\right)$$
    $$= \Sum^m_{k=0}\binom{m}{k}\left(f^{(k)}(x)g^{(m+1-k)}(x) + f^{(k+1)}(x)g^{(m+1-(k+1))}(x)\right)$$
    $$= \binom{m}{0}f^{(0)}(x)g^{(m+1-0)}(x) + 
        \Sum^m_{k=1}\binom{m}{k}f^{(k)}(x)g^{(m+1-k)}(x) + \binom{m}{k+1}f^{(k)}(x)g^{(m+1-k)}$$
    $$= \binom{m+1}{0}f^{(0)}(x)g^{(m+1-0)}(x) + \Sum^m_{k=1}\binom{m+1}{k}f^{(k)}(x)g^{(m+1-k)}(x)$$
    $$= \Sum^{m+1}_{k=1}\binom{m+1}{k}f^{(k)}(x)g^{(m+1-k)}(x)$$
\end{solution}
\vspace{6pt}
\hrule
\vspace{6pt}
% ---------------------------------------------------------

% Copy this section again for a new solution
% ---------------------------------------------------------
\ChapterHeader{2}{11}
Prove that $(1+x)^n-1\geq nx$ using mathematical induction
\begin{solution} \hfill \\
    $$(1+x)^{m+1} - 1$$
    $$(1+x)(1+x)^m -1$$
    $$1(1+x)^m+x(1+x)^m -1$$
    Inductive hypothesis is $(1+x)^m \geq mx$\\
    $$\geq x(1+x)^m+mx1$$
    $$\geq (m+(1+x)^m)x1$$
    $\forall x \geq 0, (1+x)^m > 1$
    $$\geq (m+1)x1$$
\end{solution}
\vspace{6pt}
\hrule
\vspace{6pt}
% ---------------------------------------------------------

% Copy this section again for a new solution
% ---------------------------------------------------------
\ExerciseHeader{Sage}{}
Sieve of Eranthoses 
\begin{solution} \hfill \\
    \begin{algo}
    def sieve\_of\_eranthoses(n):\+
    \\  \# All integers in the range [2, n)
    \\  numbers = list(range(2, n))
    \\  primes = []
    \\  \# Keep sifting until the current number is \geq \sqrt{n}
    \\  while numbers[0] < sqrt(n):\+
    \\      current\_prime = numbers[0]
    \\      primes.append(current\_prime)
    \\      \# some fancy python stuff here
    \\      \# filter will remove any elements that cause the provided function to return false
    \\      \# the lambda thing is a fancy way of writing $f(x) = \neg current\_prime|x$
    \\      numbers = list(filter(lambda x: x\%current\_prime != 0, numbers))\-
    \\  primes += numbers
    \\  return primes
    \end{algo}
\end{solution}
\vspace{6pt}
\hrule
\vspace{6pt}
% ---------------------------------------------------------

\end{document}

