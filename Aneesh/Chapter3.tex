% ---------
%  Compile with "pdflatex hw0".
% --------
%!TEX TS-program = pdflatex
%!TEX encoding = UTF-8 Unicode

\documentclass[11pt]{article}
\usepackage{jeffe,handout,graphicx}
\usepackage[utf8]{inputenc}     % Allow some non-ASCII Unicode in source
\usepackage{amsmath}
\usepackage{listings}
\usepackage{tikz}

%  Redefine suits
\usepackage{pifont}
\def\Spade{\text{\ding{171}}}
\def\Heart{\text{\textcolor{Red}{\ding{170}}}}
\def\Diamond{\text{\textcolor{Red}{\ding{169}}}}
\def\Club{\text{\ding{168}}}

\def\Cdot{\mathbin{\text{\normalfont \textbullet}}}
\def\Sym#1{\textbf{\texttt{\color{BrickRed}#1}}}
\let\oriinfty=\infty
\def\infty{{\oriinfty\llap{\vrule height.24em depth-.2em width.2em}}}

%%% magic code starts
\mathcode`*=\string"8000
\begingroup
\catcode`*=\active
\xdef*{\noexpand\textup{\string*}}
\endgroup
%%% magic code ends
%%% check mark
\def\checkmark{\tikz\fill[scale=0.4](0,.35) -- (.25,0) -- (1,.7) -- (.25,.15) -- cycle;}

% =====================================================
%   Define common stuff for solution headers
% =====================================================
\Class{Abstract Algebra}
\Semester{}
\Authors{1}
\AuthorOne{[Your name here]}
%\Section{}

% =====================================================
\begin{document}

% Copy this section again for a new solution
% ---------------------------------------------------------
\TheoremHeader{3}{23} % replace Chapter with Exercise for a different header
\begin{solution} \hfill \\
In a group $G$, the following laws hold ($\forall g,h\in G$):
\begin{itemize}
    \item[a)] $g^mg^n = g^{m+n} \quad \forall m,n\in \mathbb{Z}$
    \item[b)] $\left(g^m\right)^n = g^{mn} \quad \forall m,n\in \mathbb{Z}$
    \item[c)] $(gh)^n = \left(h^{-1}g^{-1}\right)^{-n} \quad \forall n\in \mathbb{Z}$\\
    Furthermore, if $G$ is abelian:
    \begin{itemize}
        \item[1)] $(gh)^n=g^nh^n$
    \end{itemize}
\end{itemize}
Proof:\\
\begin{itemize}
    \item[a)]\\
    For a given $m\in \mathbb{Z}$, $g^mg=g^{m+1}$ by the definition of exponent.\\
    Assume that for all $k=1,\dots,n$, $g^mg^k=g^{m+k}$\\
    For $k=n+1$:\\
    $$g^mg^{n+1}=g^mg^ng$$
    Applying the inductive hypothesis:
    $$=g^{m+n}g=g^{m+n+1}$$
    This proves $a$ for $n\in \mathbb{Z}^{+}$, similar logic can be used to prove for $n\in \mathbb{Z}^{-}$. To prove that this hold for a fixed $n$ and arbitrary $m$, a similar strategy can be used and the result from this proof can be applied to simplify the working.
    \item[b)]Use the result from $a$ $n$ times if $n$ is positive, otherwise prove inductively?
    \item[c)]Prove inductively that $(gh)^n(h^{-1}g^{-1})^n=1$ then multiply by the inverse of $(h^{-1}g^{-1})^n$ while applying $b$
    \item[c.1)]
\end{itemize}
\end{solution}
\vspace{6pt}
\hrule
\vspace{6pt}
% ---------------------------------------------------------

% Copy this section again for a new solution
% ---------------------------------------------------------
\ExerciseHeader{3}{4} % replace Chapter with Exercise for a different header
\begin{solution} \hfill \\
\begin{itemize}
    \item[1)] \begin{itemize}
        \item[1.] $3\textrm{ }(mod\textrm{ }7)$
    \end{itemize}
    \item[2)] \begin{itemize}
        \item[a.] No. There is no identity
        \item[b.] Yes. Let $a=0, b=1,c=2,d=3$ and $\circ = \oplus$ which is a known group
        \item[c.] Yes. This is polymorphic to $(\mathbb{Z}_4, +)$ which is a known group  
        \item[d.] No. $a$ is the identity, but not every element has an inverse.
    \end{itemize}
    \item[7)] $*$ is commutative:
    $$a*b = a+b+ab=b+a+ab=b+a+ba=b*a$$
    $0$ is the identity:
    $$a*0=a+0+(a\cdot 0)=0$$
    Need to show that every element has an inverse. We can solve $a*b=0$ for a fixed $a$.
    $$a+b+ab=0$$
    $$b(1+a)=-a$$
    $$b=\frac{-a}{(1+a)}$$
    $\forall a\in\mathbb{R}\setminus \{-1\}\quad \frac{-a}{(1+a)} \textrm{ exists} $\\
    Hence, $(S,*)$ is an abelian group.
\end{itemize}
\end{solution}
\vspace{6pt}
\hrule
\vspace{6pt}
% ---------------------------------------------------------

\end{document}

